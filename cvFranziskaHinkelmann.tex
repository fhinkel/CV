  \documentclass[10pt]{article}

  % This is a helpful package that puts math inside length specifications
  \usepackage{calc}

  % Layout: Puts the section titles on left side of page
  \reversemarginpar

  %
  %         PAPER SIZE, PAGE NUMBER, AND DOCUMENT LAYOUT NOTES:
  %
  % The next \usepackage line changes the layout for CV style section
  % headings as marginal notes. It also sets up the paper size as either
  % letter or A4. By default, letter was used. If A4 paper is desired,
  % comment out the letterpaper lines and uncomment the a4paper lines.
  %
  % As you can see, the margin widths and section title widths can be
  % easily adjusted.
  %
  % ALSO: Notice that the includefoot option can be commented OUT in order
  % to put the PAGE NUMBER *IN* the bottom margin. This will make the
  % effective text area larger.
  %
  % IF YOU WISH TO REMOVE THE ``of LASTPAGE'' next to each page number,
  % see the note about the +LP and -LP lines below. Comment out the +LP
  % and uncomment the -LP.
  %
  % IF YOU WISH TO REMOVE PAGE NUMBERS, be sure that the includefoot line
  % is uncommented and ALSO uncomment the \pagestyle{empty} a few lines
  % below.
  %

  %% Use these lines for letter-sized paper
  \usepackage[paper=letterpaper,
              %includefoot, % Uncomment to put page number above margin
              %marginparwidth=1.2in,     % Length of section titles
              %marginparsep=.05in,       % Space between titles and text
              margin=1in,
              hmargin={.5in, 1in},               % 1 inch margins
              includemp]{geometry}

  %% Use these lines for A4-sized paper
  %\usepackage[paper=a4paper,
  %            %includefoot, % Uncomment to put page number above margin
  %            marginparwidth=30.5mm,    % Length of section titles
  %            marginparsep=1.5mm,       % Space between titles and text
  %            margin=25mm,              % 25mm margins
  %            includemp]{geometry}

  %% More layout: Get rid of indenting throughout entire document
  \setlength{\parindent}{0in}

  %% This gives us fun enumeration environments. compactenum will be nice.
  \usepackage{paralist}

  %% Reference the last page in the page number
  %
  % NOTE: comment the +LP line and uncomment the -LP line to have page
  %       numbers without the ``of ##'' last page reference)
  %
  % NOTE: uncomment the \pagestyle{empty} line to get rid of all page
  %       numbers (make sure includefoot is commented out above)
  %
  \usepackage{fancyhdr,lastpage}
  \pagestyle{fancy}
  %\pagestyle{empty}      % Uncomment this to get rid of page numbers
  \fancyhf{}
  \fancyhead{}
  \fancyhead[LO,LE]{\large \bf Dr. Franziska B. Hinkelmann} % left, odd, even 
  \fancyheadoffset[L]{1in} % let left header go over margin


  %\renewcommand{\headrulewidth}{0.1 pt}

  \fancyfootoffset{\marginparsep+\marginparwidth}
  \newlength{\footpageshift}
  \setlength{\footpageshift}
            {0.5\textwidth+0.5\marginparsep+0.5\marginparwidth-2in}
  \lfoot{\hspace{\footpageshift}%
         \parbox{4in}{\, \hfill %
                      \arabic{page} of \protect\pageref*{LastPage} % +LP
  %                    \arabic{page}                               % -LP
                      \hfill \,}}

  % Finally, give us PDF bookmarks
  \usepackage{color,hyperref}
  \definecolor{darkblue}{rgb}{0.0,0.0,0.3}
  \hypersetup{colorlinks,breaklinks,
              linkcolor=darkblue,urlcolor=darkblue,
              anchorcolor=darkblue,citecolor=darkblue}

  %%%%%%%%%%%%%%%%%%%%%%%% End Document Setup %%%%%%%%%%%%%%%%%%%%%%%%%%%%


  %%%%%%%%%%%%%%%%%%%%%%%%%%% Helper Commands %%%%%%%%%%%%%%%%%%%%%%%%%%%%

  % The title (name) with a horizontal rule under it
  %
  % Usage: \makeheading{name}
  %
  % Place at top of document. It should be the first thing.
  \newcommand{\makeheading}[1]%
          {\hspace*{-\marginparsep minus \marginparwidth}%
           \begin{minipage}[t]{\textwidth+\marginparwidth+\marginparsep}%
                  {\large \bfseries #1}\\[-0.15\baselineskip]%
                  %{\large \bfseries #1}\\[-0.15\baselineskip]%
                   \rule{\columnwidth}{1pt}%
           \end{minipage}}

  % The section headings
  %
  % Usage: \section{section name}
  %
  % Follow this section IMMEDIATELY with the first line of the section
  % text. Do not put whitespace in between. That is, do this:
  %
  %       \section{My Information}
  %       Here is my information.
  %
  % and NOT this:
  %
  %       \section{My Information}
  %
  %       Here is my information.
  %
  % Otherwise the top of the section header will not line up with the top
  % of the section. Of course, using a single comment character (%) on
  % empty lines allows for the function of the first example with the
  % readability of the second example.
  \renewcommand{\section}[2]%
          {\pagebreak[2]\vspace{1.3\baselineskip}%
           \phantomsection\addcontentsline{toc}{section}{#1}%
           \hspace{-.05in}%
  	\marginpar{\raggedright \scshape #1} #2}
  %         { \marginpar
  %         \raggedright \scshape #1}#2}
  %{\large \scshape #1 } } #2}
  %{\raggedright \large \scshape #1 } \nolinebreak } #2}

  % An itemize-style list with lots of space between items
  \newenvironment{outerlist}[1][\enskip\textbullet]%
          {\begin{enumerate}[#1]}{\end{enumerate}%
           \vspace{-.6\baselineskip}}

  % An itemize-style list with little space between items
  \newenvironment{innerlist}[1][\enskip\textbullet]%
          {\begin{compactenum}[#1]}{\end{compactenum}}

  %\newenvironment{tabularr}[1][\enskip]%
  \newenvironment{tabularr}[1]%
  	{\begin{tabular}{ p{.7in}p{\textwidth -1.5in}}#1}{\end{tabular}}

  % To add some paragraph space between lines.
  % This also tells LaTeX to preferably break a page on one of these gaps
  % if there is a needed pagebreak nearby.
  \newcommand{\blankline}{\quad\pagebreak[2]}

  %%%%%%%%%%%%%%%%%%%%%%%% End Helper Commands %%%%%%%%%%%%%%%%%%%%%%%%%%%

  %%%%%%%%%%%%%%%%%%%%%%%%% Begin CV Document %%%%%%%%%%%%%%%%%%%%%%%%%%%%

  \begin{document}
  %\makeheading{Dr. Franziska B. Hinkelmann}
  %
  % NOTE: Mind where the & separators and \\ breaks are in the following
  %       table.
  %
  % ALSO: \rcollength is the width of the right column of the table 
  %       (adjust it to your liking; default is 1.85in).
  %
  \newlength{\rcollength}\setlength{\rcollength}{2.5in}%
  %
  \begin{tabular}[t]{@{}p{\textwidth-\rcollength}p{\rcollength}}
  Software Consultant & 
  	\href{mailto:franziska.hinkelmann@gmail.com}{franziska.hinkelmann@gmail.com}\\
  %TNG Technology Consulting GmbH & 
  %	\\%\textit{Phone:} (540) 257-3992\\
  %Beta-Str. 13a 
  Munich, Germany
  & 
  %85774 Unterf\"ohring, Germany 

  	GitHub: \href{https://github.com/fhinkel}{github.com/fhinkel}
  	%\href{http://people.mbi.ohio-state.edu/hinkelmann.1/}{people.mbi.ohio-state.edu/hinkelmann.1/}\\
  \end{tabular}\\

  \begin{tabular*}{\textwidth}{c}
  \hline
  \end{tabular*}

  %-----------------------------------
  \section{} 
  \textbf{TNG Technology Consulting GmbH, Unterf\"ohring, Germany}
  \begin{outerlist}
  	\item[]
  	{\bf Software Consultant}, since September 2013

    Media and Telecommunication Service Provider:
    \begin{itemize}
      \item[] Refactoring frontend and business logic (PHP)
      \item[] Contributing to rules engine (PHP) based on Rete Algorithm
      \item[] Conception and development of REST client (PHP)
      \item[] Coordinating product department, QA, and external agencies for planning and development of monthly releases
      \item[] Testdriven development with PHPUnit, object oriented programming with PHP 5.3 and PHP 5.5, Composer, Phing, PhpStorm
      \item[] Setup continuous integration pipeline with Jenkins
      \item[] Development of acceptance tests in JUnit for product information service (Java)
      \item[] Contributing to REST service for external partners (Zend Framework)
    \end{itemize}

    Real Estate Startup:
      \begin{itemize}
      \item[] Supervising student trainee
      \item[] Frontend: Angular, Backend: Node.js and MongoDB
    \end{itemize}


    Co-organizer PHP Usergroup Munich
    \begin{itemize}
      \item[] Bimonthly meetings
      \item[] 400 members
    \end{itemize}

    Conducting job interviews\\

    \end{outerlist}

  \textbf{Mathematical Biosciences Institute, Ohio
  State University, OH}
  \begin{outerlist}
  	\item[]
  	{\bf Postdoctoral Fellow}, 2011-2013\\
  	Mentor: Michael Stillman, Cornell University
  	\\
  \end{outerlist}
  %
  \begin{tabular*}{\textwidth}{c}
  \hline
  \end{tabular*}

  %-----------------------------------
  \section{Education}
  %
  \textbf{Virginia Tech, Blacksburg, VA}
  \begin{outerlist}
  \item[] 
  	{\bf Ph.D. Mathematics}, August 2011\\
  	Reinhard Laubenbacher (Virginia Bioinformatics Institute), Advisor\\
  	{\it Algebraic theory for discrete models in systems biology}
  	%Terry Herdman, Abdul Jarrah, John Burns, committee members
  \item[] 
  	{\bf M.S. Mathematics}, May 2007\\
  	%Master's Presentation: \textit{Heat Transfer in a Layered Longeron}\\
  	%Terry Herdman (Interdisciplinary Center for Applied Mathematics), Advisor\\
   	%John Burns, Gene Cliff, Ed Green, committee members
  \end{outerlist}

  \textbf{Universit\"at Karlsruhe, Karlsruhe, Germany}
  \begin{outerlist}
  \item[]
  	{\bf Vordiplom Mathematik}, minor {\bf Computer Science}, May 2006
  %\item[] 
  %	Studium Informationswirtschaft (computer
  %	science, economics, law), 2002 - 2004
  \\
  \end{outerlist}
  %\blankline
  %
  %\textbf{Markgrafen Gymnasium Karlsruhe, Karlsruhe, Germany}
  %\begin{outerlist}
  %\item[]
  %  Abitur 2002
  %\end{outerlist}
  %
  \begin{tabular*}{\textwidth}{c}
  \hline 
  \end{tabular*}

  %-----------------------------------
  \section{Opensource Contributions}
  %
  Main contributor to BlitzPay, a cryptocurrency based payment app. Winner at Burda Hackathon {\it Future of Finance} of the special prize for the highest economic impact by the Bavarian Ministry of Economic Affairs and Media, Energy and Technology, 2015\\ 

  Main contributor to open source project in Node.js: Web interface for research tool for algebraic geometry, used in courses at Harvard, Cornell, and UC Berkeley, \href{http://web.macaulay2.com}{web.macaulay2.com}\\

  Contributor to \href{https://github.com/padraic/mockery}{Mockery}, PHP mock object framework, 2014\\

  \begin{tabular*}{\textwidth}{c}
  \hline 
  \end{tabular*}

  %-----------------------------------
  \section{Publications}
  %
  \textit{A Web Application for Macaulay2}, L. Kastner, {\bf F. Hinkelmann}, M. Stillman, 2015,
  under review\\

  \textit{Steady state analysis of Boolean molecular
  network models via model reduction and
  computational algebra}, A. Veliz-Cuba, B. Aguilar, {\bf F. Hinkelmann}, R. Laubenbacher, 
  BMC Bioinformatics, 2014, \href{http://www.biomedcentral.com/1471-2105/12/295}{DOI: 10.1186/1471-2105-15-221}\\
  %
  %\textit{Downregulation of LRP6 inhibits growth of melanoma cells}, {\bf F.
  %  Hinkelmann}, B. Delidow, R. Laubenbacher, 2012, under review\\

  \textit{Inferring Biologically Relevant Models: Nested Canalyzing Functions},
  {\bf F. Hinkelmann}, A. Jarrah, ISRN Biomathematics, 2012, \href{http://www.isrn.com/journals/biomathematics/2012/613174/cta/}{DOI: 10.5402/2012/613174}\\
    
  \textit{ADAM: Analysis of Discrete Models of Biological Systems Using Computer
  Algebra}, {\bf F. Hinkelmann}, M. Brandon, B. Guang, R. McNeill,
  G. Blekherman, A. Veliz-Cuba, R. Laubenbacher, BMC Bioinformatics, 2011, \href{http://dx.doi.org/10.1186/1471-2105-12-295}{DOI: 10.1186/1471-2105-12-295}\\

  \textit{\href{http://arxiv.org/abs/1010.2669}{Fast
  Gr\"obner Basis Computation for Boolean Polynomials}},
  \textbf{F. Hinkelmann}, E. Arnold, 2010, \href{http://arxiv.org/abs/1010.2669}{arXiv.org}\\

  \textit{A Mathematical Framework for Agent Based Models of Complex Biological
   Networks},
  {\bf F. Hinkelmann}, D. Murrugarra, A. Jarrah, R. Laubenbacher,
  Bulletin of Mathematical Biology, 2010,
  \href{http://www.springerlink.com/content/7155270x6j61232w/}{DOI: 10.1007/s11538-010-9582-8}\\

  \textit{Parameter estimation for Boolean models of biological networks},
  E. Dimitrova, L. Garc\'ia-Puente, {\bf F. Hinkelmann}, A. Jarrah, R. Laubenbacher, B. Stigler,
  M. Stillman, P. Vera-Licona, Journal of Theoretical Computer Science, April
  2010, \href{http://dx.doi.org/10.1016/j.tcs.2010.04.034}{DOI: 10.1016/j.tcs.2010.04.034}\\

  \textit{Boolean Models of Bistable
  Biological Systems}, {\bf F. Hinkelmann}, R.
  Laubenbacher, Journal of Discrete and Continuous Dynamical
  Systems - Series S (DCDS-S) 4-6 December 2011 special issue on Biomathematics, \href{http://dx.doi.org/10.3934/dcdss.2011.4.1443}{DOI: 10.3934/dcdss.2011.4.1443}\\

  \begin{tabular*}{\textwidth}{c}
  \hline 
  \end{tabular*}

  %-----------------------------------
  \section{Book Chapters}
  \textit{Algebraic models and their use in systems biology}, 
      R. Laubenbacher, {\bf F. Hinkelmann}, D. Murrugarra, A. Veliz-Cuba, Discrete and Topological Models in Molecular Biology, edited by Natasa Jonoska and Masahico Saito, Springer, ISBN: 9783642401923, 2013\\
      
  	\textit{Agent-based models and optimal control in biology: an algebraic	approach}, {\bf F. Hinkelmann}, M. Oremland, R. Laubenbacher, Mathematical Concepts and Methods in Modern Biology, Edited by Raina Robeva and Terrell Hodge, Elsevier, ISBN: 9780124157804, 2013\\
  	
  	\textit{Finite Fields in Biology}, {\bf F. Hinkelmann}, R. Laubenbacher, Handbook on Finite Fields, edited by Gary Mullen and Daniel Panario, CRC Press, ISBN: 9781439873786, 2013\\

  \begin{tabular*}{\textwidth}{c}
  \hline 
  \end{tabular*}

  %-----------------------------------
  \section{Research Grants}
  %
  Collaborator on \href{http://www.nsf.gov/awardsearch/showAward.do?AwardNumber=1146819&WT.z_pims_id=5444}{NSF Award \#1146819} \textit{Collaborative Research: ABI Innovation: PlantSimLab: A Simulation Laboratory for Plant Biology}, awarded amount \$881,510, 2012\\

  \begin{tabular*}{\textwidth}{c}
  \hline 
  \end{tabular*}

  %-----------------------------------
  \section{Invited Talks}
  \textbf{JSConf EU}, \textit{JavaScript engines}, September 25-27, 2015, Berlin, Germany\\
  %

  \textbf{Nordic.js}, \textit{A Trip to the Zoo (JavaScript engines)}, September 10-11, 2015, Stockholm, Sweden\\

  \textbf{.concat(), the web development conference in Austria}, \textit{Mobile Web Apps with Native App Features}, March 7, 2015, Salzburg, Austria\\

  \textbf{PHPBenelux Conference}, Workshop \textit{From nightmare legacy code to a professional PHP application in 3 hours}, January 23, 2015, Antwerp, Belgium\\

%  \textbf{The National Alliance for Doctoral Studies in the Mathematical Sciences}, Field of Dreams Conference, November 2, 2012, Tempe, AZ\\
%
%  \textbf{MBI (Mathematical Biosciences Institute)}, Plenary Talk, Workshop: Algebraic Methods in Evolutionary and Systems Biology, May 2012, Columbus, OH \\
%
%  \textbf{Colorado State University}, Applied Mathematics Seminar, April 4, 2012, Fort Collins, CO\\
%
%  \textbf{Virginia Tech, Department of Mathematics}, Colloquium, \textit{Algebraic theory for discrete models in systems biology}, March 16, 2012, Blacksburg, VA \\
%
%  \textbf{AWM (Association for Women in Mathematics) Workshop} in conjunction with the Joint Mathematics Meeting, January 7, 2012, Boston, MA\\
%
%  \textbf{SACNAS} (Society for Advancing Hispanics/Chicanos and Native Americans
%  in Science),  Discrete Systems Biology: Unlocking Nature's Secrets One Step at a Time, October, 2011, San Jose, CA\\
%
%  \textbf{Georgia Tech}, Mathematical Biology Seminar, September 21, 2011, Atlanta, GA\\
%
%  \textbf{Clemson University}, Algebraic Geometry Seminar, \textit{Parameter
%  Estimation for ``biologically relevant" polynomial dynamical systems}, March
%  17, 2011, SC\\
%
%  \textbf{Duke University}, Algebraic Geometry Seminar, \textit{Analysis of Discrete Models of Biological Systems Using Computer Algebra}, January 26, 2011, Durham, NC \\
%
%  \textbf{North Carolina State University}, Symbolic Computation Seminar, \textit{Analysis of Discrete Models of Biological Systems Using Computer Algebra}, 
%  December 1, 2010, Raleigh, NC \\
%
%  \textbf{Karlsruhe Institute of Technology}, Seminar Institut f\"ur Wissenschaftliches Rechnen und Mathematischer Modellbildung), \textit{Algebraic
%  Varieties in Systems Biology}, November 5, 2010, %Karlsruhe, 
%  Germany
%  \\
%
  \begin{tabular*}{\textwidth}{c}
  \hline 
  \end{tabular*}

  %-----------------------------------
  % \section{Workshops Organized}
  % %
  % \textbf{MBI (Mathematical Biosciences Institute)} Workshop for Young Researchers 
  % in Mathematical Biology (WYRMB), organized jointly with R. Leander, September 2012, Columbus, OH\\

  % \textbf{Wake Forest University, School of Medicine}
  % Workshop together with R. Laubenbacher, 
  % \textit{\href{http://admg.vbi.vt.edu/home/Outreach/Workshops/cancer-systems-biology/Worksheet.pdf?attredirects=0&d=1}{Mathematical Modeling in Cancer Biology}}, March 25, 2011,
  % Winston-Salem, NC  \\

  % \textbf{Math Institutes Modern Math Workshop} Mini-course together with 
  % R. Laubenbacher, 
  % \textit{\href{http://admg.vbi.vt.edu/programs-conferences/mathematics-and-the-systems-biology-of-cancer-course-material/SystemsBiology.pdf?attredirects=0&d=1}{Mathematics
  % and the Systems Biology of Cancer}}, September 29-30, 2010,
  % Anaheim, CA \\


  % \begin{tabular*}{\textwidth}{c}
  % \hline 
  % \end{tabular*}

  % %
  % %-----------------------------------
%  \section{Mentoring}
%  %
%  \textbf{Team Investigation}, May 2013, Columbus, OH\\
%    Led a team of five undergraduate students over the course of two weeks, investigating a publication in 
%    mathematical biology\\
%    
%  \textbf{Mentor} {\bf Research Experiences for Undergraduates (REU)},
%    \href{http://biomath.vbi.vt.edu/}{\textit{Modeling and Simulation in Systems
%    Biology (MSSB)}}, Summer 2012, Blacksburg, VA\\
%    Chris Miles, Emily Petty, {\it Sensitivity Analysis for Polynomial Dynamical Systems}\\
%
%  \textbf{Mentor at Young Mathematicians Conference}, Ohio State University, August 19 - 21, 2011, Columbus, OH\\
%  	
%  \textbf{Mentor REU},
%    \href{http://biomath.vbi.vt.edu/}{\textit{MSSB}}, Summer 2011, Blacksburg, VA
%  	\begin{itemize}
%  	    \item Led a group of four undergraduates during a ten week project on {\it Optimal Control for Polynomial Dynamical Systems} and {\it Translating complex Agent Based Models into Polynomial Dynamical Systems}
%  		\item Played large role in conceptual design of the project
%  		\item Student presented at {\it Young Mathematicians Conference} at Ohio State University, August, 2011, Columbus, OH
%  		\item Outstanding Presentation award winner at the 2012 Joint Mathematics Meetings MAA
%  		Undergraduate Poster Session, {\it Heuristic Optimal Control on
%  		Polynomial Dynamical Systems Expedited by Use of Algebraic Geometry} \\
%  	  \end{itemize}
%  	
%   \textbf{Mentor} for Undergraduate Research, \textit{Mathematical Modeling
%    for Biologists, Knockout and Knockdown}, Spring 2011, Blacksburg, VA \\
%
%  \textbf{Mentor} for Undergraduate Research, \textit{Database for Discrete Models of Biological Models}, Fall 2010, Blacksburg, VA\\
%
%
%  	\textbf{Mentor} {\bf REU}, \href{http://biomath.vbi.vt.edu/}{\textit{Modeling and Simulation in Systems Biology (MSSB)}}, Summer 2010, Blacksburg, VA
%    \begin{itemize}
%
%      \item Resulted in publication: \textit{ADAM: Analysis of Discrete Models of Biological Systems Using Computer
%  	Algebra}, {\bf F. Hinkelmann}, M. Brandon, B. Guang, R. McNeill,
%  	G. Blekherman, A. Veliz-Cuba, R. Laubenbacher, BMC Bioinformatics, 2011, \href{http://dx.doi.org/10.1186/1471-2105-12-295}{DOI: 10.1186/1471-2105-12-295}
%  	%\item Students presented at {\it MAA Undergraduate Poster Session}, Joint Mathematics Meetings, New Orleans, LA, January 8, 2011\\ 
%    \end{itemize}
%
%  	\textbf{Mentor} for {\bf Initiative for Maximizing Student Development (IMSD)} Undergraduate Research, 
%  	\textit{Network Modeling}, Spring and Summer 2009, Blacksburg, VA \\
%
%
%  	\textbf{Mentor} for Undergraduate Research, 
%  	\textit{Network Modeling}, Fall 2008, Blacksburg, VA
%  %	\begin{itemize}
%  %		\item Mentored an individual student in her undergraduate research\\
%  		%\item Defined the project and set research goals 
%  		%\item Weekly meetings. 
%  		%\item Final report
%  %	\end{itemize}
%
%
%  \begin{tabular*}{\textwidth}{c}
%  \hline
%  \end{tabular*}
%

  %-----------------------------------
  \section{Presentations}
%
 % \textbf{TNG Summer Retreat} Workshop \textit{TNG in the community}, July 24, 2015, Seefeld, Austria\\
%
  \textbf{TNG Winter Retreat} Workshop \textit{Stack machines with PHP}, March 13, 2015, Seefeld, Austria\\

  \textbf{TNG Techday} Workshop \textit{Spa{\ss} mit Node.js}, February 13, 2015, Unterf\"ohring, Germany\\

  % \textbf{Plant Immunity: Pathways and Translation}, Keystone Symposium, {\it 
  % PlantSimLab Software Demonstration}, April 9, 2013, Big Sky, MT\\

  % \textbf{Plant Immunity: Pathways and Translation}, Keystone Symposium, Poster {\it PlantSimLab: a simulations laboratory for plant biology}, April 9, 2013, Big Sky, MT\\

  % \textbf{MBI (Mathematical Biosciences Institute)}, Postdoctoral seminar, November 8, 2012, Columbus, OH\\

  % \textbf{Society for Mathematical Biology}, Annual Meeting, Session for {\it Agent-based Models of Biological Systems: Approximation and Control}, {\it Algebraic Framework for Agent-based models}, July 26, 2012, Knoxville, TN\\

  % \textbf{Ohio State University}, RUMBA Undergraduate Seminar in Mathematical Biology, {\it Inhibiting tumor growth of melanoma cells}, April 3, 2012, Columbus, OH\\

  % \textbf{MBI}, Institute Partner Meeting, Poster {\it Using Systems Biology Approach to Identify Potential Targets for Therapy in Melanoma cells}, February 12, 2012, Columbus, OH\\

  % \textbf{NIMBioS (National Institute for Mathematical and Biological Synthesis)}, {\it Optimal Control on Polynomial Dynamical Systems Expedited by Use of Algebraic Geometry}, Working Group Agent-Based Models of Biological Systems, December 13, 2011, Knoxville, TN\\

  % \textbf{Ohio Wesleyan University}, Science Lecture Series, {\it Discrete Models in Systems Biology}, November 3, 2011, Delaware, OH\\

  % \textbf{Ohio Wesleyan University}, Mathematics Colloquium, {\it Algebraic theory for discrete models in systems biology}, November 3, 2011, Delaware, OH\\

  % \textbf{Center for Mathematical Medicine at Nationwide Children's Hospital}, Seminar, September 29, 2011, Columbus, OH\\

  % \textbf{MBI}, Postdoctoral Seminar, {\it Algebraic theory for discrete models in systems biology}, September 8, 2011, Columbus, OH\\

  % \textbf{Young Mathematicians Conference}, Ohio State University, {\it Panel: Graduate School Orientation Session}, August 21, 2011, Columbus, OH\\

  % \textbf{Virginia Bioinformatics Institute, PhD defense}
  % \textit{Algebraic theory for discrete models in systems biology}, August 1, 2011, Blacksburg, VA\\

  % \textbf{Virginia Tech Department of Mathematics, Graduate Seminar}
  % \textit{Writing a good CV, research statement, and teaching statement}, April
  % 7, 2011, Blacksburg, VA\\

  % \textbf{Virginia Tech Department Graduate Student Assembly, Research
  % Symposium}
  % \textit{Algebraic Framework for Discrete Models in Systems Biology}, March
  % 23, 2011, Blacksburg, VA\\

  % \textbf{Virginia Tech Department of Mathematics, Visitors' Day}
  % \textit{Polynomial Dynamical Systems in Cancer Biology}, March 18, 2011,
  % Blacksburg, VA\\

  % \textbf{Universit\"at G\"ottingen, Macaulay2 Workshop} \textit{Parameter
  % Estimation for ``Biologically Relevant'' Polynomial Dynamical Systems}, March
  % 3, 2011, G\"ottingen, Germany\\

  % \textbf{Virginia Tech Department of Mathematics, Research Days}
  % \textit{Mathematics of Systems Biology}, October 29, 2010, Blacksburg, VA \\

  % \textbf{Virginia Tech Student Chapter of SIAM, Student Research Seminars}
  % \textit{Algebraic Varieties in Systems Biology}, October 20, 2010, Blacksburg, VA \\

  % \textbf{MBI} (Mathematical Biosciences Institute), Workshop for Young Researchers 
  % in Mathematical Biology (WYRMB), Poster \textit{ADAM: Analysis of Discrete
  % Models of Biological Systems Using Computer Algebra}, August 30-September 1,
  % 2010, Columbus, OH\\

  % \textbf{SIAM} (Society for Industrial and Applied Mathematics)
  % Southeastern-Atlantic Section Conference \textit{Mathematical Framework for
  % Agent Based Models and Optimal Control}, Mathematical Modeling in Life
  % Sciences: Control and Optimization, Part 2, March 2010, Raleigh, NC\\

  % \textbf{AMS} (American Mathematical Society) \textbf{Joint Mathematics Meeting} 
  % Panel Discussion \textit{Finding a Research Topic and Thesis Advisor}, MAA Committee on Graduate
  % Students and the Young Mathematicians' Network, January, 2010, San Francisco,
  % CA\\

  % \textbf{SAMSI} (Statistical and Applied Mathematical Sciences Institute) 
  % Working Group \textit{Random Delay Networks}, February
  % 4, 2009, Research Triangle Park, NC\\

  % \textbf{AMS Joint Mathematics Meeting} \textit{A General Method to derive
  % a Boolean Model from a Continuous Model for Gene Regulatory Networks}, January
  % 6, 2009, Washington, DC\\

  % \textbf{Virginia Tech Physics Department, Graduate Student Seminar}
  % \textit{Gene Regulation in the Lac Operon}, September 12, 2008, Blacksburg, VA\\

  % \textbf{Virginia Tech Student Chapter of SIAM, Student Research Seminars}
  % \textit{General Method
  % to Transition from a Continuous Model to a Discrete Network}, September 3,
  % 2008, Blacksburg, VA\\

  % \textbf{SIAM Annual Meeting} \textit{Understanding Dynamics of
  % Gene Regulation Using a Discrete Model}, July 7-11, 2008, San Diego, CA\\
  %
  \begin{tabular*}{\textwidth}{c}
  \hline
  \end{tabular*}

  %-----------------------------------
  \section{Honors}
  \textbf{AWM Travel Award} AWM Workshop in conjunction with the Joint Mathematics Meeting, January 4-7, 2012, Boston, MA\\

  \textbf{MBI Postdoctoral Fellow} 2011-2014\\

  \textbf{Steeneck Fellowship} 2010-2011\\

  \textbf{AMS Travel Award} Joint
    Mathematics Meeting, January 13-16, 2010, San Francisco, California \\

  \textbf{SACNAS Travel Scholarship} (Advancing Hispanics/Chicanos and Native Americans
  in Science), Improving the Human Condition: Challenges for Interdisciplinary
  Science, October 15-18, 2009, Dallas, Texas \\

  \textbf{SIAM Travel Award} SIAM Annual Meeting, July 6-10, 2009, Denver, Colorado \\

  \textbf{SIAM Student Chapter Certificate of Recognition} 2009, faculty advisor Lizette Zietsman \\

  \textbf{SIAM Student Travel Award} SIAM Annual Meeting, July
  7-11, 2008, San Diego, California \\

  \textbf{Hatcher Fellowship} Summer 2008 \\

  \textbf{Baden-W\"urttemberg Stipendium} Scholarship 2006-2007 \\
  %\textbf{Travel Support}
  %\begin{outerlist}
  %  \item[] \textbf{NIMBioS} Optimal Control and Optimization for
  %  Individual-based and Agent-based Models, December 1-3, 2009, Knoxville,
  %  Tennessee

  %  \item[] \textbf{DIMACS} (Center for Discrete Mathematics and Theoretical
  %  Computer Science) Polynome working group, June 27-30, 2009, Rutgers, New Jersey
  %	\item[] \textbf{SAMSI} 
  %	Software Workshop Polynome working group, February 24-26, 2009
  %	\item[] \textbf{AMS Joint Mathematics Meeting}, January 5-8, 2009
  %	\item[] \textbf{SAMSI} Discrete Models in Systems Biology Workshop, December 3-5, 2008
  %	\item[] \textbf{SAMSI} Opening Workshop Program on Algebraic
  %	Methods in Systems Biology and Statistics, September 14-17,
  %	2008
  %	\item[] \textbf{IMA} (Institute for Mathematics and Its Applications)
  %	Workshop Design Principles in Biological
  %	Systems, April 21-25, 2008, Minneapolis, Minnesota
  %	\item[] \textbf{IMA} 
  %	Tutorial Network Dynamics and Cell Physiology, April 17-18, 2008
  %	\item[] \textbf{IMA} 
  %	Workshop Organization of Biological
  %	Networks, March 3-7, 2008\\
  %\end{outerlist}

  \begin{tabular*}{\textwidth}{c}
  \hline
  \end{tabular*}

  %-----------------------------------

  \section{Selected Workshops}
  %
  \textbf{Kanban Training}, January 24, 2014, Unterf\"ohring, Germany\\

  \textbf{Macaulay2 Workshop}, January 6 - 10, 2014, Berkeley, California \\

  \textbf{SQL Workshop}, December 13, 2013, Unterf\"ohring, Germany\\

  % Working Group \textbf{Agent-Based Models of Biological Systems: Pathways to control and optimization}, NIMBioS (National Institute for Mathematical and Biological Synthesis), November 27-29, 2012, Knoxville, TN\\

  % MBI BioSciences Problem-Solving Workshop, \textbf{Mathematical model of blood calcium ion regulation}, July 16-20, 2012, Columbus, OH\\

  % Working Group \textbf{Agent-Based Models of Biological Systems: Pathways to control and optimization}, NIMBioS (National Institute for Mathematical and Biological Synthesis), December 13-15, 2011, Knoxville, TN\\

  % \textbf{Macaulay2 Workshop}, Institute for Mathematics and Its Applications, July 25-29, 2011, Minneapolis, MN\\

  % \textbf{Macaulay2 Workshop}, Courant Center for Higher Order Structures, February 28 - March 5, 2011, G\"ottingen, Germany \\

  % \textbf{AIM} (American Institute of Mathematics) \textbf{Parameter
  % identification in graphical models}, October 4-8, 2010, Paolo Alto,
  % California \\

  % \textbf{MBI} \textbf{Bootcamp in Cancer Modeling}, September 7-10, 2010,
  % Columbus, Ohio\\

  % \textbf{Macaulay2 Workshop}, August 7-13, 2010, Colorado Springs, Colorado\\

  % \textbf{Macaulay2
  %   Workshop}, January 7 - 13, 2010, Berkeley, California \\

  % \textbf{MSRI} \textbf{Graduate
  % Workshop} Toric Varieties, June 15-26, 2009, Berkeley, California \\

  \begin{tabular*}{\textwidth}{c}
  \hline
  \end{tabular*}

  %-----------------------------------

  \section{Association Memberships}
  %\textbf{Society for Industrial and Applied Mathematics (SIAM)}, member since 2008\\
  %
  %\blankline
  %
\textbf{Grace Hopper Celebration of Women in Computing Conference (GHC)}
  \begin{outerlist}
  \item[] 
    \textbf{Poster Committee member}, 2015\\
  \end{outerlist}

%  \textbf{SIAM Student Chapter at Virginia Tech}
%  \begin{outerlist}
%  \item[] 
%  	\textbf{President}, 2008 - 2009\\
%  	Organized Student Research Seminar, visiting speakers
%  	program, field trip to Joint Mathematics Conference,
%  	Minisymposia, website maintenance\\
%  \end{outerlist}
%
  %\blankline
  %
  %%\textbf{Memberships}
  %\textbf{Pi Mu Epsilon}
  %\begin{outerlist}
  %\item[] 
  %	Mathematics Honorary Member, since 2007\\
  %\end{outerlist}
  %
  \begin{tabular*}{\textwidth}{c}
  \hline
  \end{tabular*}
  %\newpage
  %-----------------------------------

  \section{Teaching Experience}
  %
  \textbf{Universit\"at Bremen, Bremen, Germany}
  \begin{outerlist}
    \item[]
    {\bf Lecturer}, August 2015\\
   \href{https://www.informatica-feminale.de/Sommer2015/lib/ajax/course.php?courseId=562}{Agile Software Development: A Node.js application in one week}  \\
   2 Credit Points (ECTS)\\
    \item[]
    {\bf Lecturer}, August 2014\\
   \href{https://www.informatica-feminale.de/Archiv/Sommerstudium/Sommer2014/Agil.php#AGI03}{Agile Software Development: A Node.js application in one week}  \\
      2 Credit Points (ECTS)\\

  \end{outerlist}
%
%  \textbf{Virginia Tech}
%  \begin{outerlist}
%  \item[] 
%  	\textbf{Course Instructor} - Integral Calculus, Summer 2009 
%  	\begin{itemize}
%  		 \item Lectured five times a week for a class of 32 undergraduate students
%  		 \item Created syllabus
%           \item Wrote and graded course exams
%  		 \item Assigned grades to students
%  	\end{itemize}
%  \item[] 
%  	\textbf{Course Instructor} - Computer-tested Integral
%  	Calculus, Spring 2009 
  %	\begin{itemize}
  %		 \item Lectured three times a week for a class of 80 undergraduate students
  %		 \item Held office hours twice a week
  %		 \item Assigned grades to students
  %	\end{itemize}
%  \item[] 
%  	\textbf{Course Instructor} - Differential Calculus, Fall
%  	2007 - Fall 2008
  %	\begin{itemize}
  %		\item Lectured two or three times a week for a class of 40 undergraduate students 
  %		\item Created quizzes and worksheets
  %		\item Graded homework and \textit{Mathematica} worksheets 
  %		\item Held office hours three times a week
  %		\item Wrote and contributed to the creation of course exams
  %		\item Held review sessions for exams
  %		\item Graded exams
  %		\item Assigned grades to students 
  %	\end{itemize}
%  \item[]
%  	\textbf{Instructional Assistant}, Fall 2006 
%  	\begin{itemize}
%  		\item Assisted students with math concepts and software questions 
%  	\end{itemize}
  %	for a variety of courses and software such as differential calculus, integral 
  %	calculus, elementary linear algebra, business calculus, Microsoft Excel, 
  %	MATLAB, and Mathematica.
 % \end{outerlist}
%
%  \blankline
%
%  \textbf{Universit\"at Karlsruhe}
%  \begin{outerlist}
%  \item[] 
%  	\textbf{Numerical Analysis Programming Lab Instructor},
%  	Spring 2006
%  	\begin{itemize}
%  		\item Instructed students once a week and graded C++ and Java exercises
%  	\end{itemize}
%  \item[] 
%  	\textbf{Linear Algebra and Real Analysis Recitations}, Spring 2005 - Spring 2006
%  	\begin{itemize}
%  		\item Lectured once a week for a class of 70 undergraduate students
%  		\item Graded homework and exams
%  	\end{itemize}
%  \end{outerlist}
%
%  \blankline

  \begin{tabular*}{\textwidth}{c}
  \hline
  \end{tabular*}

  		
  % \section{Other Work Experience}
  % \textbf{EnBW Trading GmbH}, Karlsruhe, Germany, Summer 2007
  % \begin{outerlist}
  % \item[] 
  % 	\textbf{Internship Methods and Models at EnBW Trading GmbH}
  % 	\begin{itemize}
  % 		\item Research on energy consumption and production in Europe.	
  % 	\end{itemize}
  % \end{outerlist}

  % \blankline

  % \textbf{United Internet AG}, Karlsruhe, Germany, 2001 - 2006
  % \begin{outerlist}
  % \item[] 
  % 	\textbf{Unix System Developer}
  % 	\begin{itemize}
  % 		\item Responsible for planning, programming and testing of software,
  % 		mainly written in C, C++
  % 		%\item Extensive use of shell scripting and common Unix tools
  % 		\item Projects included Voice over IP server software and 
  % 		low level storage software.  
  % 	\end{itemize}
  % \end{outerlist}

  % \blankline

  % \begin{tabular*}{\textwidth}{c}
  % \hline
  % \end{tabular*}

  % \section{Programming Experience}
  %   Common UNIX scripting languages\\
  % 
  %   Ruby on Rails\\
  %   \textit{
  %   \href{http://polymath.vbi.vt.edu:3000/}{http://polymath.vbi.vt.edu/polynome/}}\\
  %   
  %   C, C++\\
  %   \textit{\href{http://www.math.uiuc.edu/Macaulay2/doc/Macaulay2-1.4/share/doc/Macaulay2/BooleanGB/html/}{BooleanGB -- Compute Gr\"obner Basis in a Boolean
  %   Ring}}\\
  %   
  %   Test Driven Development, Unit testing\\
  % 
  %   Computational Algebra Systems (Macaulay2, Singular)\\
  %   \textit{\href{http://www.math.vt.edu/people/fhinkel/ncf.lib}{Inferring
  %   Biologically Relevant Models: Nested Canalyzing Functions}}\\
  %   
  %   Unix Server Administration, for \textit{Applied Discrete
  %   Mathematics Group} at Virginia Bioinformatics Institute \\
  % 
  %   Common web development tools, HTML, JavaScript, jQuery, Perl, Ruby\\
  %   \textit{ADAM, Analysis of Dynamic Algebraic Models}\\
  %   \textit{TryM2!
  %   An interactive website to try Macaulay2. In progress}\\
  % 
  %   NetLogo, a multi-agent programmable modeling environment for agent-based
  %   models
  % 
  % \blankline
  % 
  % \begin{tabular*}{\textwidth}{c}
  % \hline
  % \end{tabular*}
  \section{Languages}
  \noindent
  Fluent: German (Native), English, Italian\\
  Basic: French, Spanish\\

  \begin{tabular*}{\textwidth}{c}
  \hline
  \end{tabular*}
  		
  \section{References}
  \noindent Reinhard Laubenbacher, \href{mailto:reinhard@vbi.vt.edu}{reinhard@vbi.vt.edu}, Virginia Tech, Virginia Bioinformatics Institute,\\ \noindent PhD advisor\\

  Mike Stillman, \href{mailto:mike@math.cornell.edu}{mike@math.cornell.edu}, Cornell University, Department of Mathematics\\
  %
  %Bernd Sturmfels, \href{mailt:bernd@math.berkeley.edu}{bernd@math.berkeley.edu}, UC Berkeley, Department of Mathematics\\
  %
  %Eileen Shugart, \href{mailto:shugart@vt.edu}{shugart@vt.edu}, Virginia Tech, Department of Mathematics, concerns Teaching\\

  \end{document}

  %%%%%%%%%%%%%%%%%%%%%%%%%% End CV Document %%%%%%%%%%%%%%%%%%%%%%%%%%%%%

