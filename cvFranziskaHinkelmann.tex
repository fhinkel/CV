  \documentclass[10pt]{article}

  % This is a helpful package that puts math inside length specifications
  \usepackage{calc}

  % Layout: Puts the section titles on left side of page
  \reversemarginpar

  %
  %         PAPER SIZE, PAGE NUMBER, AND DOCUMENT LAYOUT NOTES:
  %
  % The next \usepackage line changes the layout for CV style section
  % headings as marginal notes. It also sets up the paper size as either
  % letter or A4. By default, letter was used. If A4 paper is desired,
  % comment out the letterpaper lines and uncomment the a4paper lines.
  %
  % As you can see, the margin widths and section title widths can be
  % easily adjusted.
  %
  % ALSO: Notice that the includefoot option can be commented OUT in order
  % to put the PAGE NUMBER *IN* the bottom margin. This will make the
  % effective text area larger.
  %
  % IF YOU WISH TO REMOVE THE ``of LASTPAGE'' next to each page number,
  % see the note about the +LP and -LP lines below. Comment out the +LP
  % and uncomment the -LP.
  %
  % IF YOU WISH TO REMOVE PAGE NUMBERS, be sure that the includefoot line
  % is uncommented and ALSO uncomment the \pagestyle{empty} a few lines
  % below.
  %

  %% Use these lines for letter-sized paper
  \usepackage[paper=letterpaper,
              %includefoot, % Uncomment to put page number above margin
              %marginparwidth=1.2in,     % Length of section titles
              %marginparsep=.05in,       % Space between titles and text
              margin=1in,
              hmargin={.5in, 1in},               % 1 inch margins
              includemp]{geometry}

  %% Use these lines for A4-sized paper
  %\usepackage[paper=a4paper,
  %            %includefoot, % Uncomment to put page number above margin
  %            marginparwidth=30.5mm,    % Length of section titles
  %            marginparsep=1.5mm,       % Space between titles and text
  %            margin=25mm,              % 25mm margins
  %            includemp]{geometry}

  %% More layout: Get rid of indenting throughout entire document
  \setlength{\parindent}{0in}

  %% This gives us fun enumeration environments. compactenum will be nice.
  \usepackage{paralist}

  %% Reference the last page in the page number
  %
  % NOTE: comment the +LP line and uncomment the -LP line to have page
  %       numbers without the ``of ##'' last page reference)
  %
  % NOTE: uncomment the \pagestyle{empty} line to get rid of all page
  %       numbers (make sure includefoot is commented out above)
  %
  \usepackage{fancyhdr,lastpage}
  \pagestyle{fancy}
  %\pagestyle{empty}      % Uncomment this to get rid of page numbers
  \fancyhf{}
  \fancyhead{}
  \fancyhead[LO,LE]{\large \bf Dr. Franziska B. Hinkelmann} % left, odd, even 
  \fancyhead[RO,RE]{Compiler engineer, PhD in Mathematics} % right, odd, even 
  \fancyheadoffset[L]{1in} % let left header go over margin

  %\renewcommand{\headrulewidth}{0.1 pt}

  \fancyfootoffset{\marginparsep+\marginparwidth}
  \newlength{\footpageshift}
  \setlength{\footpageshift}
            {0.5\textwidth+0.5\marginparsep+0.5\marginparwidth-2in}
  \lfoot{\hspace{\footpageshift}%
         \parbox{4in}{\, \hfill %
                      \arabic{page} of \protect\pageref*{LastPage} % +LP
  %                    \arabic{page}                               % -LP
                      \hfill \,}}

  % Finally, give us PDF bookmarks
  \usepackage{color,hyperref}
  \definecolor{darkblue}{rgb}{0.0,0.0,0.3}
  \hypersetup{colorlinks,breaklinks,
              linkcolor=darkblue,urlcolor=darkblue,
              anchorcolor=darkblue,citecolor=darkblue}

  %%%%%%%%%%%%%%%%%%%%%%%% End Document Setup %%%%%%%%%%%%%%%%%%%%%%%%%%%%


  %%%%%%%%%%%%%%%%%%%%%%%%%%% Helper Commands %%%%%%%%%%%%%%%%%%%%%%%%%%%%

  % The title (name) with a horizontal rule under it
  %
  % Usage: \makeheading{name}
  %
  % Place at top of document. It should be the first thing.
  \newcommand{\makeheading}[1]%
          {\hspace*{-\marginparsep minus \marginparwidth}%
           \begin{minipage}[t]{\textwidth+\marginparwidth+\marginparsep}%
                  {\large \bfseries #1}\\[-0.15\baselineskip]%
                  %{\large \bfseries #1}\\[-0.15\baselineskip]%
                   \rule{\columnwidth}{1pt}%
           \end{minipage}}

  % The section headings
  %
  % Usage: \section{section name}
  %
  % Follow this section IMMEDIATELY with the first line of the section
  % text. Do not put whitespace in between. That is, do this:
  %
  %       \section{My Information}
  %       Here is my information.
  %
  % and NOT this:
  %
  %       \section{My Information}
  %
  %       Here is my information.
  %
  % Otherwise the top of the section header will not line up with the top
  % of the section. Of course, using a single comment character (%) on
  % empty lines allows for the function of the first example with the
  % readability of the second example.
  \renewcommand{\section}[2]%
          {\pagebreak[2]\vspace{1.3\baselineskip}%
           \phantomsection\addcontentsline{toc}{section}{#1}%
           \hspace{-.05in}%
  	\marginpar{\raggedright \scshape #1} #2}
  %         { \marginpar
  %         \raggedright \scshape #1}#2}
  %{\large \scshape #1 } } #2}
  %{\raggedright \large \scshape #1 } \nolinebreak } #2}

  % An itemize-style list with lots of space between items
  \newenvironment{outerlist}[1][\enskip\textbullet]%
          {\begin{enumerate}[#1]}{\end{enumerate}%
           \vspace{-.6\baselineskip}}

  % An itemize-style list with little space between items
  \newenvironment{innerlist}[1][\enskip\textbullet]%
          {\begin{compactenum}[#1]}{\end{compactenum}}

  %\newenvironment{tabularr}[1][\enskip]%
  \newenvironment{tabularr}[1]%
  	{\begin{tabular}{ p{.7in}p{\textwidth -1.5in}}#1}{\end{tabular}}

  % To add some paragraph space between lines.
  % This also tells LaTeX to preferably break a page on one of these gaps
  % if there is a needed pagebreak nearby.
  \newcommand{\blankline}{\quad\pagebreak[2]}
  
\newcommand{\CC}{C\nolinebreak\hspace{-.05em}\raisebox{.4ex}{\tiny\bf +}\nolinebreak\hspace{-.10em}\raisebox{.4ex}{\tiny\bf +}}
\def\CC{{C\nolinebreak[4]\hspace{-.05em}\raisebox{.4ex}{\tiny\bf ++}}}


  %%%%%%%%%%%%%%%%%%%%%%%% End Helper Commands %%%%%%%%%%%%%%%%%%%%%%%%%%%

  %%%%%%%%%%%%%%%%%%%%%%%%% Begin CV Document %%%%%%%%%%%%%%%%%%%%%%%%%%%%

  \begin{document}
  %\makeheading{Dr. Franziska B. Hinkelmann}
  %
  % NOTE: Mind where the & separators and \\ breaks are in the following
  %       table.
  %
  % ALSO: \rcollength is the width of the right column of the table 
  %       (adjust it to your liking; default is 1.85in).
  %
  \newlength{\rcollength}\setlength{\rcollength}{2.5in}%
  %
  \begin{tabular}[t]{@{}p{\textwidth-\rcollength}p{\rcollength}}
  Software Engineer & 
  	franziska.hinkelmann@gmail.com\\
  \href{https://medium.com/@fhinkel}{medium.com/@fhinkel}
  & 
      \href{https://github.com/fhinkel}{github.com/fhinkel}
  \end{tabular}\\

  \begin{tabular*}{\textwidth}{c}
  \hline
  \end{tabular*}
  
  %-----------------------------------
  \section{} 
    \textbf{Google, Munich, Germany, May 2016 - now}
  \begin{outerlist}
  	\item[]
  	{Software Engineer, V8, Chrome's open source high-performance JavaScript engine, \CC}
	\begin{itemize}
	\item Runtime Type Information in V8
	\item Performance optimizations
	\item Anything related to Node.js
	\end{itemize}
	\item[]
  	{Node.js Technical Steering Committee (TSC) member and Node.js collaborator}\\
	The TSC is responsible for high-level guidance of the Node.js project.\\
    \end{outerlist}
    
  \textbf{TNG Technology Consulting GmbH, Unterf\"ohring, Germany, September 2013 - April 2016}
  \begin{outerlist}
  	\item[]
  	{Senior Consultant, Java}\\
    \end{outerlist}

  \textbf{Mathematical Biosciences Institute, Ohio
  State University, OH, September 2011 - August 2013}
  \begin{outerlist}
  	\item[]
  	{National Science Foundation Postdoctoral Fellow}\\
  	Mentor: Michael Stillman, Cornell University
  	\\
  \end{outerlist}
  %
  \begin{tabular*}{\textwidth}{c}
  \hline
  \end{tabular*}

  %-----------------------------------
  \section{Education}
  %
  \textbf{Virginia Tech, Blacksburg, VA, August 2006 - August 2011}
  \begin{outerlist}
  \item[] 
  	{\bf Ph.D. Mathematics}, August 2011\\
  	Reinhard Laubenbacher (Virginia Bioinformatics Institute), Advisor\\
  	{\it Algebraic theory for discrete models in systems biology}
  \item[] 
  	{\bf M.S. Mathematics}, May 2007\\
  \end{outerlist}

  \textbf{Universit\"at Karlsruhe, Karlsruhe, Germany, October 2002 - July 2006}
  \begin{outerlist}
  \item[]
  	{\bf Vordiplom Mathematik}, minor {\bf Computer Science}, May 2006
  \\
  \end{outerlist}
  \begin{tabular*}{\textwidth}{c}
  \hline 
  \end{tabular*}
  
  %-----------------------------------
  \section{Selected Invited Talks}
  %
  Keynote \textbf{Node Interactive},  \textit{V8 and Node.js}, October 4-6, 2017, Vancouver, Canada\\

  \href{https://youtu.be/p-iiEDtpy6I}{\textbf{JSConf EU}, \textit{JavaScript engines - how do they even?}}, May 6-7, 2017, Berlin, Germany\\
  
    \href{https://www.youtube.com/watch?v=B9igDWV5ZUg}{\textbf{JSConf.Asia}, \textit{Performance Profiling for V8}}, November 26, 2016, Singapore, Singapore\\
       
  \textbf{NodeConf Barcelona}, \textit{V8 under the Hood}, November 21, 2015, Barcelona, Spain\\

 \href{https://www.youtube.com/watch?v=1kAkGWJZ6Zo}{\textbf{Nordic.js}, \textit{A Trip to the Zoo (JavaScript engines)}}, September 10-11, 2015, Stockholm, Sweden\\

  \begin{tabular*}{\textwidth}{c}
  \hline 
  \end{tabular*}

  
  \end{document}

  %%%%%%%%%%%%%%%%%%%%%%%%%% End CV Document %%%%%%%%%%%%%%%%%%%%%%%%%%%%%

