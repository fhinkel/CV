  \documentclass[10pt]{article}

  % This is a helpful package that puts math inside length specifications
  \usepackage{calc}

  % Layout: Puts the section titles on left side of page
  \reversemarginpar

  %
  %         PAPER SIZE, PAGE NUMBER, AND DOCUMENT LAYOUT NOTES:
  %
  % The next \usepackage line changes the layout for CV style section
  % headings as marginal notes. It also sets up the paper size as either
  % letter or A4. By default, letter was used. If A4 paper is desired,
  % comment out the letterpaper lines and uncomment the a4paper lines.
  %
  % As you can see, the margin widths and section title widths can be
  % easily adjusted.
  %
  % ALSO: Notice that the includefoot option can be commented OUT in order
  % to put the PAGE NUMBER *IN* the bottom margin. This will make the
  % effective text area larger.
  %
  % IF YOU WISH TO REMOVE THE ``of LASTPAGE'' next to each page number,
  % see the note about the +LP and -LP lines below. Comment out the +LP
  % and uncomment the -LP.
  %
  % IF YOU WISH TO REMOVE PAGE NUMBERS, be sure that the includefoot line
  % is uncommented and ALSO uncomment the \pagestyle{empty} a few lines
  % below.
  %

  %% Use these lines for letter-sized paper
  \usepackage[paper=letterpaper,
              %includefoot, % Uncomment to put page number above margin
              %marginparwidth=1.2in,     % Length of section titles
              %marginparsep=.05in,       % Space between titles and text
              margin=1in,
              hmargin={.5in, 1in},               % 1 inch margins
              includemp]{geometry}

  %% Use these lines for A4-sized paper
  %\usepackage[paper=a4paper,
  %            %includefoot, % Uncomment to put page number above margin
  %            marginparwidth=30.5mm,    % Length of section titles
  %            marginparsep=1.5mm,       % Space between titles and text
  %            margin=25mm,              % 25mm margins
  %            includemp]{geometry}

  %% More layout: Get rid of indenting throughout entire document
  \setlength{\parindent}{0in}

  %% This gives us fun enumeration environments. compactenum will be nice.
  \usepackage{paralist}

  %% Reference the last page in the page number
  %
  % NOTE: comment the +LP line and uncomment the -LP line to have page
  %       numbers without the ``of ##'' last page reference)
  %
  % NOTE: uncomment the \pagestyle{empty} line to get rid of all page
  %       numbers (make sure includefoot is commented out above)
  %
  \usepackage{fancyhdr,lastpage}
  \pagestyle{fancy}
  %\pagestyle{empty}      % Uncomment this to get rid of page numbers
  \fancyhf{}
  \fancyhead{}
  \fancyhead[LO,LE]{\large \bf Dr. Franziska B. Hinkelmann} % left, odd, even 
  \fancyheadoffset[L]{1in} % let left header go over margin


  %\renewcommand{\headrulewidth}{0.1 pt}

  \fancyfootoffset{\marginparsep+\marginparwidth}
  \newlength{\footpageshift}
  \setlength{\footpageshift}
            {0.5\textwidth+0.5\marginparsep+0.5\marginparwidth-2in}
  \lfoot{\hspace{\footpageshift}%
         \parbox{4in}{\, \hfill %
                      \arabic{page} of \protect\pageref*{LastPage} % +LP
  %                    \arabic{page}                               % -LP
                      \hfill \,}}

  % Finally, give us PDF bookmarks
  \usepackage{color,hyperref}
  \definecolor{darkblue}{rgb}{0.0,0.0,0.3}
  \hypersetup{colorlinks,breaklinks,
              linkcolor=darkblue,urlcolor=darkblue,
              anchorcolor=darkblue,citecolor=darkblue}

  %%%%%%%%%%%%%%%%%%%%%%%% End Document Setup %%%%%%%%%%%%%%%%%%%%%%%%%%%%


  %%%%%%%%%%%%%%%%%%%%%%%%%%% Helper Commands %%%%%%%%%%%%%%%%%%%%%%%%%%%%

  % The title (name) with a horizontal rule under it
  %
  % Usage: \makeheading{name}
  %
  % Place at top of document. It should be the first thing.
  \newcommand{\makeheading}[1]%
          {\hspace*{-\marginparsep minus \marginparwidth}%
           \begin{minipage}[t]{\textwidth+\marginparwidth+\marginparsep}%
                  {\large \bfseries #1}\\[-0.15\baselineskip]%
                  %{\large \bfseries #1}\\[-0.15\baselineskip]%
                   \rule{\columnwidth}{1pt}%
           \end{minipage}}

  % The section headings
  %
  % Usage: \section{section name}
  %
  % Follow this section IMMEDIATELY with the first line of the section
  % text. Do not put whitespace in between. That is, do this:
  %
  %       \section{My Information}
  %       Here is my information.
  %
  % and NOT this:
  %
  %       \section{My Information}
  %
  %       Here is my information.
  %
  % Otherwise the top of the section header will not line up with the top
  % of the section. Of course, using a single comment character (%) on
  % empty lines allows for the function of the first example with the
  % readability of the second example.
  \renewcommand{\section}[2]%
          {\pagebreak[2]\vspace{1.3\baselineskip}%
           \phantomsection\addcontentsline{toc}{section}{#1}%
           \hspace{-.05in}%
  	\marginpar{\raggedright \scshape #1} #2}
  %         { \marginpar
  %         \raggedright \scshape #1}#2}
  %{\large \scshape #1 } } #2}
  %{\raggedright \large \scshape #1 } \nolinebreak } #2}

  % An itemize-style list with lots of space between items
  \newenvironment{outerlist}[1][\enskip\textbullet]%
          {\begin{enumerate}[#1]}{\end{enumerate}%
           \vspace{-.6\baselineskip}}

  % An itemize-style list with little space between items
  \newenvironment{innerlist}[1][\enskip\textbullet]%
          {\begin{compactenum}[#1]}{\end{compactenum}}

  %\newenvironment{tabularr}[1][\enskip]%
  \newenvironment{tabularr}[1]%
  	{\begin{tabular}{ p{.7in}p{\textwidth -1.5in}}#1}{\end{tabular}}

  % To add some paragraph space between lines.
  % This also tells LaTeX to preferably break a page on one of these gaps
  % if there is a needed pagebreak nearby.
  \newcommand{\blankline}{\quad\pagebreak[2]}
  
\newcommand{\CC}{C\nolinebreak\hspace{-.05em}\raisebox{.4ex}{\tiny\bf +}\nolinebreak\hspace{-.10em}\raisebox{.4ex}{\tiny\bf +}}
\def\CC{{C\nolinebreak[4]\hspace{-.05em}\raisebox{.4ex}{\tiny\bf ++}}}


  %%%%%%%%%%%%%%%%%%%%%%%% End Helper Commands %%%%%%%%%%%%%%%%%%%%%%%%%%%

  %%%%%%%%%%%%%%%%%%%%%%%%% Begin CV Document %%%%%%%%%%%%%%%%%%%%%%%%%%%%

  \begin{document}
  %\makeheading{Dr. Franziska B. Hinkelmann}
  %
  % NOTE: Mind where the & separators and \\ breaks are in the following
  %       table.
  %
  % ALSO: \rcollength is the width of the right column of the table 
  %       (adjust it to your liking; default is 1.85in).
  %
  \newlength{\rcollength}\setlength{\rcollength}{2.5in}%
  %
  \begin{tabular}[t]{@{}p{\textwidth-\rcollength}p{\rcollength}}
  Software Engineer & 
  	\href{mailto:franziska.hinkelmann@gmail.com}{franziska.hinkelmann@gmail.com}\\
  %TNG Technology Consulting GmbH & 
  %	\\%\textit{Phone:} (540) 257-3992\\
  %Beta-Str. 13a 
  Munich, Germany
  & 
  %85774 Unterf\"ohring, Germany 

  	GitHub: \href{https://github.com/fhinkel}{github.com/fhinkel}
  \end{tabular}\\

  \begin{tabular*}{\textwidth}{c}
  \hline
  \end{tabular*}
  
  %-----------------------------------
  \section{} 
    \textbf{Google Germany, Munich, Germany, May 2016 - now}
  \begin{outerlist}
  	\item[]
  	{Software Engineer, V8}
	\\
\end{outerlist}
   
  \textbf{TNG Technology Consulting GmbH, Unterf\"ohring, Germany, September 2013 - March 2016}
  \begin{outerlist}
  	\item[]
  	{Senior Consultant, December 2015 - March 2016}
  	\item[]
  	{Software Consultant, September 2013 - December 2015}\\
    \end{outerlist}

  \textbf{Mathematical Biosciences Institute, The Ohio
   State University, OH, September 2011 - August 2013}
  \begin{outerlist}
  	\item[]
  	{National Science Foundation Postdoctoral Fellow}\\
  	Mentor: Michael Stillman, Cornell University
  	\\
  \end{outerlist}
  %
  \begin{tabular*}{\textwidth}{c}
  \hline
  \end{tabular*}

  %-----------------------------------
  \section{Education}
  %
  \textbf{Virginia Tech, Blacksburg, VA, August 2006 - August 2011}
  \begin{outerlist}
  \item[] 
  	{\bf Ph.D. Mathematics}, August 2011\\
  	Reinhard Laubenbacher (Virginia Bioinformatics Institute), Advisor\\
  	{\it Algebraic theory for discrete models in systems biology}
  \item[] 
  	{\bf M.S. Mathematics}, May 2007\\
  \end{outerlist}

  \textbf{Universit\"at Karlsruhe, Karlsruhe, Germany, October 2002 - July 2006}
  \begin{outerlist}
  \item[]
  	{\bf Vordiplom Mathematik}, minor {\bf Computer Science}, May 2006
  \\
  \end{outerlist}
  \begin{tabular*}{\textwidth}{c}
  \hline 
  \end{tabular*}

  %-----------------------------------
    \section{Projects}
  %
    \textbf{PHP}
      \begin{outerlist}
           \item[]
      Rules engine based on Rete Algorithm for  Media and Telecommunication Service Provider, since 2013
  \item[] 
     Contributor of \href{https://github.com/padraic/mockery}{\it Mockery}, PHP mock object framework, 2014\\
  \end{outerlist}
  
\textbf{Node.js}
      \begin{outerlist}
  \item[] 
    Main contributor to open source project {\it Interactive Shell}. Web interface for a research tool for algebraic geometry, used in courses at Harvard, Cornell, and UC Berkeley, \href{http://web.macaulay2.com}{web.macaulay2.com}, since 2011\\
  \end{outerlist}
  
\textbf{React}
    \begin{outerlist}
  \item[] 
  	Main contributor to BlitzPay, a cryptocurrency based payment app. Winner at Burda Hackathon {\it 	Future of Finance} of the special prize for the highest economic impact by the Bavarian Ministry of	Economic Affairs and Media, Energy and Technology, June 2015\\
  \end{outerlist}
  %-----------------------------------
  \begin{tabular*}{\textwidth}{c}
  \hline 
  \end{tabular*}

  %-----------------------------------
  \section{Invited Talks}
  %
  \textbf{NodeConf Barcelona}, \textit{V8 under the Hood}, November 21, 2015, Barcelona, Spain\\

  \textbf{JSConf EU}, \textit{JavaScript engines}, September 25-27, 2015, Berlin, Germany\\

  \textbf{Nordic.js}, \textit{A Trip to the Zoo (JavaScript engines)}, September 10-11, 2015, Stockholm, Sweden\\

  \textbf{.concat(), the web development conference in Austria}, \textit{Mobile Web Apps with Native App Features}, March 7, 2015, Salzburg, Austria\\

  \textbf{PHPBenelux Conference}, Workshop \textit{From nightmare legacy code to a professional PHP application in 3 hours}, January 23, 2015, Antwerp, Belgium\\

  \begin{tabular*}{\textwidth}{c}
  \hline 
  \end{tabular*}

  %----------------------------------
  \section{Workshops/ Teaching}
   \textbf{Universit\"at Bremen} Course \textit{\href{https://www.informatica-feminale.de/Sommer2015/lib/ajax/course.php?courseId=562}{Agile Software Development: A Node.js application in one week}}, 2 Credit Points (ECTS), August 17-21, 2015, Bremen, Germany\\
     
  \textbf{TNG Winter Retreat} Workshop \textit{Stack machines with PHP}, March 13, 2015, Seefeld, Austria\\

  \textbf{TNG Techday} Workshop \textit{Spa{\ss} mit Node.js}, February 13, 2015, Unterf\"ohring, Germany\\

   \textbf{Universit\"at Bremen} Course \textit{\href{https://www.informatica-feminale.de/Sommer2015/lib/ajax/course.php?courseId=562}{Agile Software Development: A Node.js application in one week}}, 2 Credit Points (ECTS), August 25-29, 2014, Bremen, Germany\\

  %
  \begin{tabular*}{\textwidth}{c}
  \hline
  \end{tabular*}

  %----------------------------------
  \section{Association Memberships}
    \textbf{Co-organizing PHP Usergroup Munich,  September 2014 - March 2015}
          \begin{outerlist}
  \item[] Member count doubled to 500 members within 1 year
  \item[] Bi-monthly meetings\\
  \end{outerlist}
  
\textbf{Grace Hopper Celebration of Women in Computing Conference (GHC)}
  \begin{outerlist}
  \item[] 
    \textbf{Poster Committee member}, 2015\\
  \end{outerlist}


  \begin{tabular*}{\textwidth}{c}
  \hline
  \end{tabular*}
  
  \end{document}

  %%%%%%%%%%%%%%%%%%%%%%%%%% End CV Document %%%%%%%%%%%%%%%%%%%%%%%%%%%%%

